\documentclass[11pt, letterpaper]{article}
\usepackage[backend=bibtex]{biblatex}
\bibliography{bib}
\usepackage[utf8]{inputenc}
\usepackage{amsmath}
\usepackage{amsfonts}
\usepackage{amssymb}
\usepackage{fullpage}
\usepackage{lmodern}
\usepackage{graphicx}
\usepackage{gensymb}
\usepackage{parskip}
\author{Revan~Sopher}
\title{Natural Language Processing Project Proposal}
\begin{document}
\maketitle

\section{Subject}
I will be attempting to build a knowledge base of ingredients, their relationships, and what one does with them by parsing recipes.
I'll then try to generate recipes based on a list of desired input ingredients.

If that falls through, I'll try suggesting ingredient replacements, where I try to find relations between ingredients based on the actions that are done to them, and on what other ingredients they occur alongside of.

I don't imagine projects in this class need much more motivation than ``it's a challenging problem''.

And of course, http://xkcd.com/720/ .

\section{Existing Work}
The ACL paper ``Cooking with Semantics'' \footnote{http://www.cs.ubc.ca/~murphyk/Papers/acl2014.pdf} provides a nice explanation of the challenges faced in parsing recipes.
Notably, subjects are often taken implicitly or indirectly available, so the recipient of an action is often unclear.

I also came across a corpus of annotated recipes \footnote{http://www.ark.cs.cmu.edu/CURD/} which might be useful for performance evaluation, and a blog post \footnote{http://datadesk.latimes.com/posts/2013/12/natural-language-processing-in-the-kitchen/} about using NLTK to parse a database of recipes (with ugly hacks to overcome domain-specific issues).

\section{Evaluation}
I may be able to use the existing annotated corpus above to judge the accuracy of the collected knowledge, which would provide a quantitative measure.
Otherwise evaluation is rather qualitative: if I am able to generate recipes, I would consider the project successful if it results in coherent steps. Culinary style is, so to speak, off the table.

\end{document}
